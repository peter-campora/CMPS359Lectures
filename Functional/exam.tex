\documentclass[11pt]{article}

%%%
%%% XeTeX:
%%%
%
%!TEX TS-program = xelatex
%\usepackage{amsmath}
%\usepackage{mathspec}
%\usepackage{xunicode}
%\usepackage{xltxtra}
%%
%\usepackage{goudy}
% \usepackage{arnopro}     % edgy italics, has small caps
% \usepackage{minionpro}   % thick, edgy italics, has small caps
% \usepackage{caslonpro}   % italics too slanted, nice k, has small caps


% \usepackage[greeklowercase=upright,mdugm]{mathdesign}

%\usepackage{MnSymbol}
%\usepackage{wasysym}
%\usepackage{multicol}
\usepackage{upquote}
\usepackage{stmaryrd}
\usepackage{lambda}
% \usepackage{ifsym}
%\usepackage{wasysym}
\usepackage{vmargin}
\usepackage{fancyvrb}
\usepackage{enumitem}
%\usepackage{wrapfig}
\usepackage{mathpartir}
\usepackage{needspace}

%\input{term.tex}

\newcommand{\term}{Fall 2019}

\sloppypar
\parindent0pt

\setpapersize{USletter}
\setmarginsrb{23mm}{23mm}{23mm}{23mm}{10mm} {5mm} {10mm} {0mm}
%%            lft   top   rght  bot   hdhgt hdsep fthgt ftsep


%%% EXAM, COURSE TITLE, AND TERM
%%%
\def\exam{Exam \#1}
\def\course{CMPS 359, \term}


%%% HEADER
%%%
\def\hdr#1{\def\hdrhd{#1}}
\makeatletter
\def\ps@hdr{\let\@mkboth\@gobbletwo
%      \def\@oddhead{\hdrhd}\def\@oddfoot{\reset@font\hfil\sl Good Luck! \hfil}
     \def\@oddhead{\hdrhd}\def\@oddfoot{}
     \let\@evenhead\@empty\let\@evenfoot\@oddfoot}
\makeatother
% \hdr{{\sl \exam, \course}\hfil\arabic{page}}
\hdr{{\textit{\exam, \course}}\hfil\arabic{page}}
\pagestyle{hdr}
\thispagestyle{plain}


%%% ENVIRONMENTS
%%%
\DefineVerbatimEnvironment{program}{Verbatim}
  {baselinestretch=1.0,%
   xleftmargin=0mm,samepage=true,commandchars=\\\{\}}

\DefineVerbatimEnvironment{programL}{Verbatim}
  {baselinestretch=1.0,%
   xleftmargin=0mm,samepage=true}

\newcounter{exerc}
\newcommand{\question}[2][{}]%
{\refstepcounter{exerc}\vspace{8ex plus2ex}%
 \noindent{\large\bf \arabic{exerc}. #2\ \hrulefill{#1}}
 \par\bigskip}



%%% MACROS
%%%
\newcommand{\prog}[1]{\texttt{#1}}

\newcommand{\OR}{\ |\ }
%\newcommand{\set}[1]{\OB{\{#1\}}}
\def\OB#1{\ifmmode#1\else\mbox{$#1$}\fi}
\def\denseitems{
    \itemsep1pt plus1pt minus1pt
    \parsep0pt plus0pt
    \parskip0pt\topsep0pt}

\newcommand{\Int}{\textit{Int}}
\newcommand{\Name}{\textit{Name}}
\newcommand{\Stmt}{\textit{Stmt}}
\newcommand{\Expr}{\textit{Expr}}
\newcommand{\Pars}{\textit{Pars}}
\newcommand{\Decl}{\textit{Decl}}
\newcommand{\Type}{\textit{Type}}

\newcommand{\xemph}[1]{\textbf{\emph{#1}}}


\begin{document}

\vspace{-2cm}

\hrule
\smallskip

\noindent
\rule[-.35mm]{\textwidth}{0.7mm}
\smallskip

\noindent
{\LARGE\textbf{\exam}\hfill \course}
\medskip

\noindent
\rule[1.05mm]{\textwidth}{0.7mm}
% \rule{\textwidth}{0.7mm} \\
% \smallskip
%
% \noindent
\hrule
% \date{}
% \maketitle

% \bigskip\bigskip\bigskip\noindent
\bigskip\bigskip\bigskip\noindent
Name/ULID:\ \dotfill\dotfill\hfill{}

%%%
%%% LEARNING OBJECTIVES
%%%
%
% QUESTION
%  1 2 3 4 5
%  x _ _ _ _  1 Abstract Syntax
%  _ _ _ _ _  2 Static & Dynamic Scoping
%  _ _ _ x _  3 Static & Dynamic Typing
%  _ _ _ _ _  4 Parameter Passing
%  _ _ _ _ _  5 Runtime Stack
%  _ _ _ _ _  6 Exception Handling
%  _ _ _ _ x  7 Polymorphism
%  _ _ x _ _  8 Paradigms
%  _ x x _ _  9 Semantics
%
% Need (2, 4, 5, and) 6 in Final
%
%%%
%%%
%%%

%\newlength{\dummylen}
% \newcommand{\WARN}[1]{}
%\newcommand{\NOTE}[1]{\setlength{\dummylen}{\fboxrule}\setlength{\fboxrule}{2pt}%
%            \vspace{1ex}\noindent%
%            \fbox{\begin{minipage}{.99\columnwidth}#1\end{minipage}}%
%            \setlength{\fboxrule}{\dummylen}\vspace{1ex}}

\newenvironment{grammar}{\par\bigskip\begin{tabular}[t]{lcl}}
                        {\end{tabular}\par\bigskip\noindent}
\newenvironment{grammarM}{\par\bigskip$\begin{array}[t]{lcl}}{\end{array}$}

\newcommand{\nt}[1]{\textit{#1}}

\vspace{5mm}

\newcommand{\WARN}[1]{\setlength{\dummylen}{\fboxrule}\setlength{\fboxrule}{1pt}%
            \vspace{1ex}\noindent%
            \fbox{\begin{minipage}{.99\columnwidth}#1\end{minipage}}%
            \setlength{\fboxrule}{\dummylen}\vspace{1ex}}
\newlength{\dummylen}

Note: 
This exam contains 6 pages. Please make sure that you have that number of pages
(plus one more page of notes).
The overall points are 100 plus 5 bonus points. 

\vspace{-6pt}

%\WARN{This exam contains 7 pages. 
%
%(1) Unless given in the question, give type signatures for your function definitions.
%
%(2) Your function definitions should be robust in the sense that they don't
%make assumptions about their arguments (for example, the length of lists) that
%are not mentioned explicitly in the exercise.
%}
%

\newcommand{\sol}{\iffalse}


\question[3*10=30 points]{Expression evaluations}

For each of the following expressions, (1) intuitively describe what
it does and (2) give the evaluation result.


\begin{enumerate}[label=(\alph*)]
\item \prog{(string-length (string-append "foo" "bar"))}

\sol
3
\else
 \vspace{12mm}
\fi


\item
\begin{program}
  (define (my-func x) (+ (sqr x) 2))
  (my-func 2)
\end{program}
\sol
3
\else
 \vspace{12mm}
\fi

\item 

\begin{program}
 (/ 2 5)
\end{program}

\sol
\begin{program}
"HSE"
\end{program}
\else
 \vspace{12mm}
\fi

\item

\begin{program}
  (struct point [x y])
  (point 1 2)
  (point-x 1)
\end{program}

\sol
\begin{program}
"akll"
\end{program}
\else
 \vspace{12mm}
\fi


\item \prog{(substring "hello" 0 3)}

\sol
\begin{program}
["functional","Haskell"]
\end{program}
\else
 \vspace{12mm}
\fi

\item \prog{(string-ref "hello" 1)}

\sol
\begin{program}
["lanoitcnuf","lleksaH"]
\end{program}
\else
 \vspace{12mm}
\fi

\item
  \begin{program}
    (define x 1)
    (define y 2)
    (cond
      [(= x 2) x]
      [(> y 1) y])
  \end{program}

\sol
\begin{program}
["lanoitcnuf!","lleksaH!"]
\end{program}
\else
 \vspace{12mm}
\fi

\item \prog{(rectangle 10 10 "black" "outline")}  draw the result
\sol
\begin{program}
[-3,-5,-4,-7]
\end{program}
\else
 \vspace{12mm}
\fi

\item \prog{(if (< 1 2) "true" "false")}

\sol
\begin{program}
"Haskell is functional"
\end{program}
\else
 \vspace{12mm}
\fi

\item \prog{(image-height (rectangle 10 10 "black" "outline"))}

\sol
\begin{program}
[8,12]
\end{program}
\else
 \vspace{12mm}
\fi

\end{enumerate}

\question[5*4=20 points]{Design Process}

Consider the following function definition corresponding to: $||(x, y, z)|| = \sqrt{x^2+y^2+z^2}$: 
\begin{program}
  (1)  (struct r3 [x y z])
  (2)  ;; An R3 is a structure
  (3)  ;; (r3 Number Number Number)
  (4)  ;; Interpretation a point in 3D real valued space
  (5)  (define ex1 (r3 1 0 0))
  (6)  (define ex2 (r3 0 2 0))
  
  (7)  ;; R3 -> Number
  (8)  ;; Calculate the distance to the origin of a point in R3
  (9)  (define (distance pt)
  (10)    (sqrt (+ (sqr (r3-x pt)) (sqr (r3-y pt)) (sqr (r3-z pt)))))
  (11) (check-expect (distance ex1) 1)
  (12) (check-expect (distance ex2) 2)
  (13) (test)
\end{program}

For each step of the design process listed below via (1) through (6), list the lines
that correspond to that step of the design process. If no lines correspond, write NA.

\begin{enumerate}[label=(\arabic*)]
\item 

\sol
\begin{program}
(2), (3), (4), (8), (11)
\end{program}
\else
 \vspace{17mm}
\fi

\item 


\sol
\begin{program}
(2), (3), (4), (8), (11)
\end{program}
\else
 \vspace{17mm}
\fi


\item 


\sol
\begin{program}
(2), (4)
\end{program}
\else
 \vspace{17mm}
\fi

\item 

\sol
\begin{program}
(10), (11)
\end{program}
\else
 \vspace{17mm}
\fi

\item

  \sol
\begin{program}
(10), (11)
\end{program}
\else
 \vspace{17mm}
\fi

\item

  \sol
\begin{program}
(10), (11)
\end{program}
\else
 \vspace{17mm}
\fi

\end{enumerate}

\question[10+5+5+10=30 points]{Carrying out Design Steps}

  \begin{enumerate}

  \item Look at the following function definition that converts
    Celsius temperatures to Fahrenheit temperatures. On line 1
    give a data interpretation for Celsius temperatures. On line 2
    provide an interpretation for Fahrenheit. On line 3 give a signature
    for the function. On line 4 give a statement of purpose. On line 7,
    turn the functional example on line 5 into a test.
    \begin{program}
      (1)
      (2)
      (3)
      (4)
      (5) Given: 0 Expect: 32
      (6) (define (C-to-F c) (+ (* 9/5 c) 32))
      (7)
    \end{program}

  \item Let us say that we then add the following test on line (8):
    \prog{(check-expect (C-to-F 5) 41)}. Does this test succeed or fail?
    If it fails, explain why it fails. Is the test incorrect or is
    the function definition incorrect?

  \item Let us say that we then add the following test on line (9):
    \prog{(check-expect (C-to-F 10) 60)}. Does this test succeed or fail?
    If it fails, explain why it fails. Is the test incorrect or is
    the function definition incorrect?

  \item Let us say we are tasked with designing a function and have finished
    steps 1-3 in our design process. The function takes in a key-event and
    returns \#t when the key that was pressed was the space bar (" "). Otherwise
    it returns \#f. Write a skeleton for the following code after line (5)
    of the top definiton. Then complete step 5 on the bottom definition and
    turn the skeleton into final code.

    \begin{program}
      (1) ;; KeyEvent -> Boolean
      (2) ;; If the spacebar was pressed return #t else return #f
      (3) ;; Given: " " Expect: #t
      (4) ;; Given: "f" Expect: #f
      (5) (define (space-press? ke)





      
    \end{program}

    \begin{program}
      (1) ;; KeyEvent -> Boolean
      (2) ;; If the spacebar was pressed return #t else return #f
      (3) ;; Given: " " Expect: #t
      (4) ;; Given: "f" Expect: #f
      (5) (define (space-press? ke)





      
    \end{program}

  \end{enumerate}
\clearpage
\question[10+10=20 points]{Defining functions}
  %%
    In these questions, you won't have to worry about the design process.
    Just write the code to complete the function definition. Assume you have
    the following defined: \prog{(struct editor [pre post])}, where the fields
      pre and post will contain strings.
\begin{enumerate}
\item Finish the following function which appends a key to the string in the pre
  of some editor struct parameter (ed) and returns some new editor struct instancecontaining this new string in pre and the editor parameter's post string in post.
  Assume this function only takes in keys that can be displayed nicely like
  "a", "!", etc. You are given:
  \begin{program}
    ;; Editor String -> Editor
    ;; Given: (editor "hello" "world") " " Expect: (editor "hello " "world")
    (define (add-key ed ke)









    
    
  \end{program}
\item Finish the following function which creates a new string that "deletes" the first character from the string in the post string of some editor parameter. This
  new string goes into the post field of some new editor that is returned. This
  new editor's pre string is the same as the editor parameter's pre string.
  \begin{program}
    ;; Editor -> Editor
    ;; Given: (editor "hello" "world") " " Expect: (editor "hello " "world")
    (define (delete-key ed)








    
    
  \end{program}
%
\end{enumerate}
\clearpage

\question[3+2=5points]
  There are two bonus point questions.

  \begin{enumerate}
  \item What is a piece of art (painting, movie, album, etc.) that you love? Why?
    \sol
    \begin{program}
      (2), (3), (4), (8), (11)
    \end{program}
    \else
    \vspace{75mm}
    \fi

  \item What do you want to do as a computer scientist? Or, if you don't plan on becoming a computer scientist, what do you want out of taking computer science courses?
    \sol
    \begin{program}
      (2), (3), (4), (8), (11)
    \end{program}
    \else
    \vspace{17mm}
    \fi
  \end{enumerate}
\clearpage






\end{document}


