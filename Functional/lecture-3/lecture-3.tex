%%% Local Variables:
%%% TeX-command-extra-options: "-shell-escape"
%%% mode: latex
%%% TeX-master: t
%%% End:
\documentclass{beamer}
\usepackage{caption}
\usepackage{minted}
\usepackage[labelformat=simple]{subcaption}

\usetheme{Singapore}
\title{Lecture 3}
%\subtitle{in Racket}
%\author{Peter Campora}
%\institute{ULL}
%\date{\today}

%This lectures introduces Racket basics and the right way to program
\begin{document}
\begin{frame}
\titlepage
\end{frame}

\section{Intro}
\begin{frame}
  \frametitle{Arithmetic Revisited}
  Let's look at some basic arithmetic:
  \begin{itemize}
  \item<2-> \mintinline{Scheme}{(+ 1 1)}
  \item<3-> Evaluate inner parentheses first: \mintinline{Scheme}{(+ 1 (+ 1 (+ 1 1) 2) 3 4 5)}
  \item<4-> == \mintinline{Scheme}{(+ 1 (+ 1 2 2) 3 4 5)}
  \item<5-> == \mintinline{Scheme}{(+ 1 5 3 4 5)}
  \item<6-> == \mintinline{Scheme}{(+ 1 5 3 4 5)}
  \item<7-> == \mintinline{Scheme}{(+ 1 (+ 1 2 2) 3 4 5)}
  \end{itemize}
\end{frame}

\begin{frame}
  \frametitle{Arithmetic Revisited (cont)}
  \begin{itemize}
  \item<1-> Primitive form is \mintinline{Scheme}{(operator [number...])}
  \item<2-> A list of useful operators:  +, -, *, /, abs, add1, ceiling, denominator, expt, floor, gcd, log, max, numerator, quotient, random, remainder, sqr, and tan
  \item<3-> Racket has an extensive \emph{numeric tower}.
  \item<4-> What would \mintinline{Scheme}{(/ 4 3)} produce in Java?
  \item<5-> Instead of producing a float, it produces $\frac{4}{3}$.
  \item<6-> To get a float, you need to use \mintinline{Scheme}{(exact->inexact (/ 4 3))}
  \item<7-> Let's play around with this a bit!
  \end{itemize}
\end{frame}

\end{document}

