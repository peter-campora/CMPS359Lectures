%%% Local Variables:
%%% TeX-command-extra-options: "-shell-escape"
%%% mode: latex
%%% TeX-master: t
%%% End:
\documentclass{beamer}
\usepackage{caption}
\usepackage{minted}
\usepackage{tikz}
\usepackage{xcolor}
\usetikzlibrary{shapes.geometric, arrows}
\tikzstyle{startstop} = [rectangle, rounded corners, minimum width=3cm, minimum height=1cm,text centered, draw=black, fill=red!30]
\tikzstyle{io} = [trapezium, trapezium left angle=70, trapezium right angle=110, minimum width=1.5cm, minimum height=0.6cm, text centered, draw=black, fill=blue!30]
\tikzstyle{process} = [rectangle, minimum width=1.5cm, minimum height=0.5cm, text centered, draw=black, fill=orange!30]
\tikzstyle{decision} = [circle, radius=2.5cm, text centered, draw=black, fill=green!30]
\tikzstyle{arrow} = [thick,->,>=stealth]
\usepackage[labelformat=simple]{subcaption}

\usetheme{Singapore}
\title{Typed Racket Guide}
\begin{document}
\begin{frame}
\titlepage
\end{frame}
\section{Typed Racket}

\begin{frame}
  \frametitle{Why Add a Type-checker to a Dynamically Typed Language?}
  Racket is a dynamically typed language like Python.
  \begin{itemize}
  \item<2-> So why add a type-checker to the language?
  \item<3-> We still design programs with type-like ideas in mind (thinking about sets of values).
  \item<4-> Types are good for documentation. When learning a new codebase
    in a dynamically typed language, we mentally reconstruct the type structure
    of the code.
  \item<5-> Typechecking can help catch errors before unit tests when code changes.
  \item<6-> Types help IDEs provide useful services.
  \end{itemize}
\end{frame}



\begin{frame}
  \frametitle{Making a Basic Typed Racket File}
  To start making a typed racket file, you  must add \mintinline{python}{\#lang typed/racket}
  \begin{itemize}
  \item<2-> If you require some module with \mintinline{python}{(require mod)} change it to
    \mintinline{python}{(require typed/mod)}
  \item<3-> Let's use an example of a program to migrate to using static types:
    %\UntypedExample
  \end{itemize}
\end{frame}

\begin{frame}
  \frametitle{Adding The Static Types}
  
\end{frame}
\end{document}
%%% Local Variables:
%%% TeX-command-extra-options: "-shell-escape"
%%% mode: latex
%%% TeX-master: t
%%% End: