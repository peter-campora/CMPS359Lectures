%%% Local Variables:
%%% TeX-command-extra-options: "-shell-escape"
%%% mode: latex
%%% TeX-master: t
%%% End:
\documentclass{beamer}
\usepackage{caption}
\usepackage{minted}
\usepackage{tikz}
\usepackage{xcolor}
\usetikzlibrary{shapes.geometric, arrows}
\tikzstyle{startstop} = [rectangle, rounded corners, minimum width=3cm, minimum height=1cm,text centered, draw=black, fill=red!30]
\tikzstyle{io} = [trapezium, trapezium left angle=70, trapezium right angle=110, minimum width=1.5cm, minimum height=0.6cm, text centered, draw=black, fill=blue!30]
\tikzstyle{process} = [rectangle, minimum width=1.5cm, minimum height=0.5cm, text centered, draw=black, fill=orange!30]
\tikzstyle{decision} = [circle, radius=2.5cm, text centered, draw=black, fill=green!30]
\tikzstyle{arrow} = [thick,->,>=stealth]
\usepackage[labelformat=simple]{subcaption}

\usetheme{Singapore}
\title{Extending Racket}

\begin{document}
\begin{frame}
\titlepage
\end{frame}

\begin{frame}
  \frametitle{A Programmable Programming Language}
  \huge Why write a program when you could write a program that writes
  programs?\\
  \begin{center}
    \includegraphics[width=0.4\textwidth]{images/fingerhead.jpg}
  \end{center}
\end{frame}

\begin{frame}
  \frametitle{Designing a Language for Extensibility}
  Racket was designed to be a \emph{programmable programming language}.
  I.E. you can add new features to the programming language or design
  entire new programming languages with seamless interoperation 
  between them.
  \begin{itemize}
  \item<2-> Racket also has some non-functional features that allow
    it to support programs from different paradigms being added on top.
  \item<3-> These include mutation, continuations, and macros
  \item<4-> Mutation is something that is common to most programming languages
    that you use.
  \item<5-> Continuations allow for the addition of things like break, continue,
    generators or coroutines.
  \item<6-> Macros allow for syntax extensions to be defined at compile time.
  \end{itemize}
\end{frame}

\defverbatim[colored]\mutate{
\begin{minted}{racket}
    (define x 2)
    (set! x 3)
    x
\end{minted}
}

\defverbatim[colored]\mutatePy{
\begin{minted}{racket}
    x = 2
    x = 3
    x
\end{minted}
}

\begin{frame}
  \frametitle{Adding Objects}
  I want to go over how we can simulate basic objects
  in Racket.
  \begin{itemize}
  \item<2-> In order to do this, I want my objects to have internal state
    that can be modified like in most OO languages.
  \item<3-> We can mutate things in Racket with functions that end with a '!'
  \item<4-> The most basic one is \mintinline{racket}{set!}
  \item<5-> \mutate
  \item<6-> This program returns 3.
  \end{itemize}
\end{frame}

\defverbatim[colored]\productPy{
  \begin{minted}{python}
    total = 1
    for x in range(1, 20):
         total *= x
    total
  \end{minted}
}

\defverbatim[colored]\product{
\begin{minted}{racket}
  (define total 1)
  (for ([x (range 1 20)])
    (set! total (* total x)))
  total
\end{minted}
}

\begin{frame}
  \frametitle{Loops and State}
  This is similar to the Python program below:
  \mutatePy
  \begin{itemize}
  \item<2-> We can even write imperative programs in Racket. Consider
    the following Python program:
  \item<3-> \productPy
  \item<4-> In Racket it is:
    \product
  \end{itemize}
\end{frame}

\defverbatim[colored]\mutateList{
\begin{minted}{racket}
  (define mut-list (mcons 1 (mcons 2 '())))
  (set-mcar! mut-list 2) ;; mutate the head
  (set-mcdr! mut-list '()) ;;muatate the tail
  mut-list ;; contains '(2)
\end{minted}
}

\begin{frame}
  \frametitle{Racket Has Your Favorite Things}
  Now you can see that you can write the programs that you are used to writing
  in Racket.
  \begin{itemize}
  \item<2-> But preferably, you'd write: \mintinline{racket}{(foldl * 1 (range 1 20))}
  \item<3-> The program that uses mutation has \emph{benign} mutation if the
    total variable isn't used in other contexts, but using a fold prevents
    mutable state from making your program hard to parallelize or test.
  \item<4-> We can also mutate things like lists and vectors.
  \item<5-> \mutateList
  \end{itemize}
\end{frame}

\defverbatim[colored]\defineVectors{
\begin{minted}{racket}
    (define vec1 #(1 2 3)) ;;immutable
    (define vec2 (vector 1 2 3))
    ;; same as vec1 but mutable
    (define vec3 (vector-immutable 1 2 3))
    ;; same as vec2 but immutable
    (define vec4 (make-vector 3 0))
    ;; same as (vector 0 0 0)
\end{minted}
}

\begin{frame}
  \frametitle{Vectors in Racket}
  One useful data structure for keeping things performant is a mutable vector.
  Think of it as being analogous to an array in Java or Python.
  \pause
  \begin{center}
    \includegraphics[width=0.4\textwidth]{images/Vector.png}
  \end{center}  
\end{frame}

\defverbatim[colored]\vecSort{
\begin{minted}{racket}
  (define b-vec (vector 3 2 1))
  (vector-sort! b-vec <)
  b-vec ;;returns #(1 2 3)
\end{minted}
}

\begin{frame}
  \frametitle{Vectors}
  Let's start with the most basic vector operations
  \begin{itemize}
  \item<2-> You can make an immutable or mutable vector with the following:
    \defineVectors
  \item<3-> We can reference an element in constant time with:
    \mintinline{racket}{(vector-ref vec1 0) ;; returns 1}
  \item<4-> We can then mutate one of the mutable vectors with:
    \mintinline{racket}{(vector-set! vec2 0 9) ;; inserts 9}
  \end{itemize}
\end{frame}

\begin{frame}
  \frametitle{Vectors}
  I want to cover a few more vector operations.
  \begin{itemize}
  \item<2-> You can think of this as the following python code:
    \mintinline{python}{vec2[0] = 9}
  \item<3-> There is also tuple assignment like the following Python code:
    \mintinline{python}{a, b, c = (1, 2, 3)}
  \item<4-> In Racket:
    \mintinline{racket}{(define-values (a b c) (vector->values vec2))}
  \item<5-> We can destructively sort mutable vectors:
    \vecSort
  \item<6-> We can destructively map with:
    \mintinline{racket}{(vector-map! add1 vec2)}
  \end{itemize}    
\end{frame}

\defverbatim[colored]\makehash{
\begin{minted}[fontsize=\footnotesize]{racket}
  (define hash1 (make-hash)) ;; empty hash
  (define hash2 (make-hash '(["a" . 1] ["b" . 2] ["c" . 3])))
  ;; this hash contains 3 keys and 3 values
  ;; The ["a" . 1] defines a pair
  ;; It is equivalent to (cons "a" 1)
\end{minted}
}

\begin{frame}
  \frametitle{Vectors and Hashes}
  So far I've focused on mutable operations on vectors, but there are
  plenty of immutable operations. But typically I reach for vectors when
  either fast access is needed or copying becomes a performance bottleneck.
  \begin{itemize}
  \item<2-> But there are other operations like
    \mintinline{racket}{vector-map}, \mintinline{racket}{vector-filter},
    \mintinline{racket}{vector-take}, \mintinline{racket}{vector-drop},
    and \mintinline{racket}{vector-length} that are useful.
  \item<3-> But for now, let's focus on a commonly used data structure,
    the hash table.
  \item<4-> \emph{Insert joke about hash here}
  \item<5-> Here are some examples of making hash tables.
  \item<6-> \makehash
  \end{itemize}
\end{frame}

\defverbatim[colored]\hashref{
\begin{minted}[fontsize=\footnotesize]{racket}
  (define hash-ex (make-hash '(["a" . 1] ["b" . 2] ["c" . 3])))
  (hash-ref hash-ex "a") ;; returns 1
  (hash-ref hash-ex "d" -1) ;; returns -1
\end{minted}
}

\defverbatim[colored]\hashset{
\begin{minted}[fontsize=\footnotesize]{racket}
  (hash-set! hash-ex "d" 4) ;; adds "d"->4 mapping
  (hash-set! hash-ex "a" -5) ;; now "a"->5 in hash
\end{minted}
}

\begin{frame}
  \frametitle{Hashing it Out}
  So, let's go over the common operations on hash tables.
  \begin{itemize}
  \item<2-> We can reference a key in the hash with
    \mintinline[fontsize=\footnotesize]{racket}{hash-ref} 
  \item<3-> \hashref
  \item<4-> In the second example I used a default value for when
    the key was not present in the hash.
  \item<5-> \hashset
  \item<6-> You can get the keys and values from a hash with
    \mintinline[fontsize=\footnotesize]{racket}{(hash-keys hash-ex)} and
    \mintinline[fontsize=\footnotesize]{racket}{(hash-values hash-ex)}, respectively.
  \item<7-> Finally, you can check if a key (say "a") is in the hash with:
    \mintinline[fontsize=\footnotesize]{racket}{(hash-has-key? hash-ex "a")}
  \end{itemize}
\end{frame}

\defverbatim[colored]\racketInterface{
\begin{minted}{racket}
  (define animal<%>
    (interface ()
      [speak (->m string?)]
      [move (->m string?)]))
\end{minted}
}

\begin{frame}
  \frametitle{Racket is an Object Oriented Language}
  Now I reveal the evil truth, that Racket is an object oriented language!
  \begin{itemize}
  \item<2-> \includegraphics[width=0.6\textwidth]{images/quirrell.jpg}
  \item<3-> It has objects, classes, interfaces, inheritance, and overriding.
  \item<4-> To start using object oriented features, we must include
    \mintinline{racket}{(require racket/class)} at the top of our file.  
  \end{itemize}
\end{frame}

\begin{frame}
  \frametitle{Writing OO Racket}
  Writing OO in Racket is syntactically heavy but doable. 
  \begin{itemize}
  \item<2-> We can define a simple interface as follows:
    \racketInterface
  \item<3-> This says that an animal is an interface that does not inherit
    from any other interfaces (hence the ()) and has one method, speak
    that returns a string.
  \item<4-> In general an interface first specifies the other interfaces it
    inherits from, followed by a list of methods that determine the structure
    of the interface.
  \item<5-> Using interfaces, we then want to have a class that implements them.
    Typically we would then design more complex classes by having many fields
    with interface types and a constructor that accepts classes that implement
    these interfaces.
  \end{itemize}
\end{frame}

\defverbatim[colored]\catClass{
\begin{minted}[fontsize=\footnotesize]{racket}
    (define cat%
      (class* object% (animal<%>)
        (super-new)
        (init name)
        (init appearance)
        (define animal-name name)
        (define animal-app appearance)
        (define/public (speak)
          (string-append "Hi, I'm " animal-name "! Meow!"))
        (define/public (move) "Walks")
        (define/public (show-off) animal-app)))
\end{minted}
}

\defverbatim[colored]\speak{
\begin{minted}{racket}
    (define/public (speak)
      (string-append "Hi, I'm " animal-name "! Meow!"))
\end{minted}
}

\begin{frame}
  \frametitle{Writing OO Racket}
    This approach to OO is called \emph{design by composition} and
    Racket is well suited for it, though you can also use a more classic
    design by inheritance approach.
    \begin{itemize}
    \item<2-> Either way, let's have a class implement our interface.
    \item<3-> \catClass
    \end{itemize}
\end{frame}

\begin{frame}
  \frametitle{Writing OO Racket}
  There was a lot to unpack in that code.
  \begin{itemize}
  \item<2-> The first few lines say that the cat class inherits/implements
    the animal interface, and its constructor starts by calling
    \mintinline{racket}{(super-new)} to call the constructor of its
    parent class, which is object.
  \item<3-> The \mintinline{racket}{(init name)} and \mintinline{racket}{(init appearance)} lines state that the constructor takes in two keyword arguments, name and appearnce.
  \item<4-> The \mintinline{racket}{(define animal-name name)}
    and \mintinline{racket}{(define animal-app appearance)} lines
    state that there are two local fields that are defined by the constructor
    arguments.
  \item<6-> A method definition implements the speak method from the animal
    interface: \speak
  \end{itemize}
\end{frame}

\defverbatim[colored]\account{
\begin{minted}[fontsize=\footnotesize]{racket}
  (define bank-account%
  (class object%
    (super-new)
    (init init-balance)
    (define balance init-balance)
    (define/public (withdraw amount)
      (cond
        [(and (< amount balance) (> amount 0))
         (println "Withdrawal accepted")
         (set! balance (- balance amount))
         amount]
        [else
         (println "Withdrawal rejected")
         0]))
    (define/public (deposit amount)
      (cond
        [(> amount 0)
         (println "Depositing amount")
         (set! balance (+ balance amount))]
        [else (println "Can't deposit negative amounts")]))
    (define/public (print-balance) (println balance))))
\end{minted}
}
\begin{frame}
  \frametitle{Bank Accounts and OO}
  \account
\end{frame}

\defverbatim[colored]\newAcc{
\begin{minted}[fontsize=\footnotesize]{racket}
    (define account-ex (new bank-account% [init-balance 100]))
\end{minted}
}

\defverbatim[colored]\deposit{
\begin{minted}[fontsize=\footnotesize]{racket}
    (send account-ex deposit 100)
\end{minted}
}

\defverbatim[colored]\withdraw{
\begin{minted}[fontsize=\footnotesize]{racket}
    (send account-ex withdraw 50)
\end{minted}
}

\defverbatim[colored]\balance{
\begin{minted}[fontsize=\footnotesize]{racket}
    (send account-ex print-balance)
\end{minted}
}

\begin{frame}
  \frametitle{Bank Accounts and OO}
  The interpretation of this is that we are defining
  a class with two methods, one for withdrawing money and one for depositing
  money. The logic is pretty straightforward. 
  \begin{itemize}
  \item<2-> I can make a new bank account with:
    \newAcc
  \item<3-> I can make a deposit with:
    \deposit
  \item<4-> I can make a withdrawal with:
    \withdraw
  \item<5-> I can check the balance with:
    \balance
  \item<6-> An interesting question is if I could still simulate something
    similar without having classes or even structs.
  \end{itemize}
\end{frame}


\defverbatim[colored]\stupid{
\begin{minted}{racket}
    (define (stupid-two)
      (define x 1)
      (lambda () (+ x 1)))
\end{minted}
}

\defverbatim[colored]\closure{
\begin{minted}{racket}
  (define x 5)
  ;; returns a lambda above
  (define two-f (stupid-two)) 
  (two-f) ;; returns 2
  ;; The x's don't clash
\end{minted}
}

\begin{frame}
  \frametitle{Closures and State}
  Let's talk about higher order functions some more.
  \begin{itemize}
  \item<2-> Consider the following function which defines a local
    variable and then returns an anonymous function
  \item<3-> \stupid
  \item<4-> The inner function that is returned has a way of remembering
    the value inside of x inside of the scope of the function. This is called
    a \emph{closure}
  \item<5-> \closure
  \end{itemize}
\end{frame}

\defverbatim[colored]\makeCounter{
\begin{minted}{racket}
    (define (make-counter)
      (define c 0)
      (lambda ()
        (set! c (+ c 1))
        c))
\end{minted}
}

\defverbatim[colored]\counter{
\begin{minted}{racket}
    (count) ;; returns 1
    (count) ;; returns 2
\end{minted}
}

\begin{frame}
  \frametitle{Closures and State}
  Ok, so saving the state of the outer variable isn't that interesting.
  Let's make something more interesting, a function that counts the number
  of times that it innns called.
  \begin{itemize}
  \item<2-> \makeCounter
  \item<3-> We can then make a function that counts like this:
    \mintinline{racket}{(define count (make-counter))}
  \item<4-> We can then call it:
    \counter
  \end{itemize}
\end{frame}

\defverbatim[colored]\bankAccountClosure{
\begin{minted}[fontsize=\footnotesize]{racket}
(define (account-closure balance)
  (define (deposit amount)
    (cond
      [(> amount 0)
       (println "Depositing")
       (set! balance (+ balance amount))]
      [else
       (println "Cannot deposit negative amounts")]))
  (define (withdraw amount)
    (cond
      [(and (> amount 0) (< amount balance))
       (println "Withdrawing")
       (set! balance (- balance amount))
       amount]
      [else
       (println "Cannot withdraw negative amounts")]))
  (define (print-balance) balance)
  (lambda (msg . args)
    (case msg
      [("deposit") (apply deposit args)]
      [("withdraw") (apply withdraw args)]
      [("print-balance") (print-balance)]
      [else (println "invalid operation")])))
\end{minted}
}

\begin{frame}
  \frametitle{Closures and State}
  \huge So now we see that a closure can encapsulate and modify state
  in a similar manner to an object. Let's try to roll a bank account
  object only using a closure. 
\end{frame}

\begin{frame}
  \bankAccountClosure
\end{frame}

\begin{frame}
  \frametitle{Closures are a Poor Man's Objects. Objects are a Poor Man's Closures.}
  \begin{itemize}
  \item<2-> There was some unintroduced magic in our bank account closure.
  \item<3-> \mintinline{racket}{(lambda (msg . args)}
  \item<4-> The . args means that we args will contain a list of a variable
    number of optional arguments. The msg parameter must always  receiv
    an argument, however.
  \item<5-> \mintinline{racket}{(case msg} defines a case expression that
    matches the value of the argument that was received in different branches.
    It is less powerful than cond but more elegant when all you are doing
    is equality checking on a variable.
  \item<6-> The heart of the logic comes from the fact that we define hidden
    nested functions that aren't available in global scope that act as our
    methods.
  \item<7-> The second pieced is that our anonymous function encapsulates
    the balance parameter for modification.
  \end{itemize}
\end{frame}
\end{document}
