\documentclass{article}
\usepackage[inline]{enumitem}
\usepackage{hyperref}
\begin{document}
The grading for projects will be done in the following way:
\begin{itemize}
\item Code that \emph{always} results in a runtime error when run, regardless of
  user interaction, will result in a 60 \emph{at best}. 
\item Code that passes all tests, including my interactions and unit tests will
  receive a 90 \emph{at worst}.
\item If your code is incomplete but does not always error when run, you will receive a 40 \emph{at worst}. But you must have \emph{some} code that looks meaningful to the assignment.
\item Code that works for some input and not others will start at a 50 and
  receive points based on the following.
\item 15 points will be assigned based on how well ``render'' works.
\item 25 points will be assigned based on how well ``edit'' works.
\item 5 points will be awarded if the design process is reflected in the code.
\item An additional 5 points will be given based on the cleanliness of the code. I.E. your code should be well commented and variable names should be meaningful.
\item This means that if edit breaks on a couple of cases, but the design process
  is clean, you can still get an A on the project.
\item This also means that a slightly erroneous submission can theoretically get a higher grade than a poorly documented and structured working submission.
\item A 100 is reserved for working programs who get all  10  aforementioned points based on the design  process and code quality.
\item Good luck! Similar criteria will be used on future projects.
\end{itemize}

\end{document}

%%% Local Variables:
%%% mode: latex
%%% TeX-master: t
%%% End:
