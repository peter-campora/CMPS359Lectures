\documentclass{article}
\usepackage[inline]{enumitem}
\usepackage{hyperref}
\begin{document}
\title{CMPS 359: Introduction to Functional Programming with Racket}
\author{Peter Campora}
%
\maketitle

\noindent Book: \href{https://htdp.org/2019-02-24/index.html}{How to Design Programs (2nd edition)}\\

\noindent The four main objectives of this course are to:
%
\begin{enumerate}
\item Learn systematic program design and problem classification.
\item Spend a nontrivial amount of time programming in a functional programming language.
\item Understand the features of functional programming languages and how the design of a functional
  program differs from an imperative program.
\item Think differently about what programming is.
\end{enumerate}

\noindent Along the way you will also learn about the following concepts:
\begin{itemize}
%\item 
\item How data and program type influences the design of programs
\item Higher Order Functions
\item Recursion
\item Immutable Data
\item and more...
%\item Employ program design as a team
\end{itemize}

\noindent Grades will come from the following sources and are weighted as follows (subject to change):
\begin{itemize}
\item Smaller Programming Assignments: 15\%
\item Projects: 50\%
\item Midterm: 15\%
\item Final Project: 20\%
\end{itemize}

\noindent Office Hours: MWF 2-4\\\\
Web Page: peter-campora.github.io
% \noindent Ideally, by the end of this course you all will pick up skills
% that are used in whatever job you get when you graduate. A secondary skill I hope
% you pick up is how to discuss the differences between programming languages
% and be able to argue about how different features affect program design.
% Finally, I hope you all become comfortable engaging in technical discussions,
% and I'd like for the class to be slightly Socratic.
\end{document}

%%% Local Variables:
%%% mode: latex
%%% TeX-master: t
%%% End:
